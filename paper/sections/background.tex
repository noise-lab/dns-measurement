\section{Background and Related Work}\label{sec:background}

In this section, we provide background on encrypted DNS protocols, including
the current deployment status of encrypted DNS, as well as various related
work on measuring encrypted DNS.

\subsection{Background: Encrypted DNS}

The Domain Name System (DNS) translates human-readable domain names into
Internet Protocol (IP) addresses, which are used to route
traffic~\cite{dns-rfcs}.  These queries have typically been unencrypted, which
enables on-path eavesdroppers to intercept queries and manipulate responses.


\paragraph{Encrypted DNS.}
Protocols for encrypting DNS traffic have been proposed, standardized, and
deployed in recent years, including DNS-over-HTTPS (DoH) and DNS-over-TLS
(DoT).  Hu et al. proposed DoT in 2016 to address the eavesdropping and
tampering of DNS queries~\cite{hu2016DoT}.  It uses a dedicated port (853)
to communicate with resolvers over a TLS connection.  In contrast,
DoH---proposed by Hoffman et al. in 2018---establishes HTTPS sessions with
resolvers over port 443~\cite{hoffman2018DoH}.  This design decision enables
DoH traffic to use HTTPS as a transport, facilitating deployment as well as
making it difficult for network operators and eavesdroppers to intercept DNS
queries and responses~\cite{boettger2019empirical}. DoH can function in many
environments where DoT is easily blocked.

\paragraph{Moving the privacy threat.} Encrypting DNS queries and responses hides queries
from eavesdroppers but the recipient of the queries---the DNS resolver---can
see the queries~\cite{IEEEfight}. By design, recursive resolvers receive
queries from clients and typically need to perform additional queries to a
series of authoritative name servers to resolve domain names.  For these
resolvers to determine the additional queries they need to perform (or
determine if the query can be answered from cache), they must be able to see
the queries that they receive from clients.  Thus, although DoT and DoH make
it difficult for eavesdroppers along an intermediate network path to see DNS
traffic, recursive resolvers can still observe (and potentially, log) the
queries that they receive from clients.  The fact that many mainstream DoH
providers (e.g., Google) already collect significant information about users
potentially raises additional privacy concerns and makes it appealing for
users to have a large number of encrypted DNS resolvers that are reliable and
perform well. For this reason, users may wish to have more control over the
recursive resolver that they use to resolve encrypted DNS queries. Having
a reasonable set of choices that perform well in the first place is thus
important, and determining whether such a set exists is the focus of this
paper.

\paragraph{Status of browser DoH deployments.}
Most major browsers currently support DoH, including Brave, Chrome, Edge,
Firefox, Opera, and Vivaldi.  Operating systems have also announced
plans to implement DoH, including iOS, MacOS, and
Windows~\cite{ffSettings,operaEdgeSettings,vivaldiSettings,iosSettings,jensen2020windows}.
In this paper, we focus on DoH because it is more widely deployed
than DoT.  Each of these browsers and operating systems either
currently support or have announced support for DoH (but not
DoT)~\cite{lack-of-dot-support}.

\subsection{Related Work}

\paragraph{Previous measurement studies of encrypted DNS.}
Previous studies have typically measured DoT and DoH response times the
protocols from the perspective of a few commonly used
resolvers~\cite{lu2019end-to-end}; in contrast, in this paper, we study a much
larger set of encrypted DNS resolvers, many of which are not available as
default options in major browsers.  Zhu et al. proposed DoT to encrypt DNS
traffic between clients and recursive resolvers~\cite{zhu2015connection}.
They modeled its performance and found that DoT's overhead can be largely
eliminated with connection re-use.  Böttger et al. measured the effect of DoT
and DoH on query response times and page load times from a university
network~\cite{boettger2019empirical}.  They find that DNS generally
outperforms DoT in response times, and DoT outperforms DoH.  They also find
that much of the performance cost for DoT and DoH can be amortized by re-using
TCP connections and TLS sessions.  Hounsel et al. also measure response times
and page load times for DNS, DoT, and DoH using Amazon EC2
instances~\cite{hounsel2020comparing}.  They compare the recursive resolvers
for Cloudflare, Google, and Quad9 to the local recursive resolvers provided by
Amazon EC2 from five global vantage points in Ohio, California, Seoul, Sydney,
and Frankfurt.  They find that despite higher response times, page load times
for DoT and DoH can be \emph{faster} than DNS on lossy networks.  Lu et al.
utilized residential TCP SOCKS networks to measure response times from 166
countries and found that, in the median case with connection re-use, DoT and
DoH were slower than conventional DNS over TCP by 9 ms and 6 ms,
respectively~\cite{lu2019end-to-end}.
In contrast to previous work, our focus in this paper is not to measure the
DoH protocol itself or its relative performance to unencrypted DNS; instead,
our goal is to compare the performance of encrypted DNS resolvers {\em to
each other}, to understand the extent to which this larger set of DNS
resolvers could be used by clients and applications in different regions of
the world.

\paragraph{Studies and remedies for the centralization of encrypted DNS.} Other work has studied the centralization
of the DNS and proposed various techniques to address it.  Foremski et al.
find that the top 10\% of DNS recursive resolvers serve approximately 50\% of
DNS traffic~\cite{foremski2019dns-observatory}.  Moura et
al.~\cite{moura2020clouding} measured DNS requests to two country code
top-level domains (ccTLD) and found that five large cloud providers being
responsible for over 30\% of all queries for the ccTLDs of the Netherlands and
New Zealand.  Hoang et al.~\cite{KResolver} developed K-resolver,
which distributes queries over multiple DoH recursive resolvers, so that no
single resolver can build a complete profile of the user and each recursive
resolver only learns a subset of domains the user resolved.  Hounsel et al.
also evaluate the performance of various query distribution strategies and
study how these strategies affect the amount of queries seen by individual
resolvers~\cite{hounsel2021multi}.
This line of research complements our work---these previous studies in many
ways motivate an enable the use of multiple encrypted DNS resolvers, but
designing a system to take advantage of multiple recursive resolvers must be
informed about how the choice of resolver affects performance.


\paragraph{Other DNS performance studies.} Researchers have also studied how
DNS performance affects application performance.  Sundaresan et al. used an
early FCC Measuring Broadband America (MBA) deployment of 4,200 home gateways
to identify performance bottlenecks for residential broadband
networks~\cite{sundaresan2013measuring}.  This study found that page load
times for users in home networks are significantly influenced by slow DNS
response times.  Allman studied conventional DNS performance from 100
residences in a neighborhood and found that only 3.6\% of connections were
blocked on DNS with lookup times greater than either 20 ms or 1\% of the
application's total transaction time~\cite{allman2020putting}. Otto et al.
found that the ability of a content delivery network to deliver fast page load
times to a client could be significantly hindered when clients choosing
recursive resolvers that are far away from CDN caches~\cite{otto2012content};
a subsequent proposal, namehelp, proxied DNS queries for CDN-hosted content
and sent them directly to authoritative servers.  Wang et al. developed WProf,
a profiling system that analyzes various factors that contribute to page load
times~\cite{wang2013demystifying}; this study demonstrated that queries for
uncached domain names at recursive resolvers can account for up to 13\% of the
critical path delay for page loads.  We do not measure the performance of a
large number of encrypted DNS resolvers from residential broadband access
networks in this paper nor do we study the effect of the choice of a broad
range of encrypted DNS resolvers on page load time, but doing so would be a
natural direction for future work.  


