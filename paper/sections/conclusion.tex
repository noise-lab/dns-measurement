\section{Conclusion}\label{sec:conclusion}

In this work, we investigated the DNS response of times of a set of global DoH
resolvers from different vantage points.  We find that although most
lesser-known resolvers have higher response times than well-known resolvers, a
select group of lesser-known resolvers had close response times, or even lower
response times, to well-known resolvers.  Additionally, we find a wide
distribution of response times between resolvers despite their close proximity
to one another. Our results suggest that while clients may generally expect the 
best DoH query response times with current major resolvers, they may also use 
certain less-popular resolvers within their geographic region. They also 
suggest that any resolver that is not majorly deployed is not useful if it 
is outside of one's region.

Previous work suggests DoH response times may not greatly matter for certain applications, such as web browsers.
Hounsel et al. found that while DoH response times were generally higher than traditional DNS, page load times were comparable~\cite{hounsel2020comparing}.
These measurements were performed with major DoH resolvers--namely Cloudflare, Google, and Quad9--but the results should generalize.
DoH connections based on HTTP/2 and above may utilize asynchronous queries, enabling the browser to send multiple DoH queries.
In short, if applications enable DoH queires to be asynchronously issued, and TCP/TLS sessions are re-used with long timeouts, then higher DNS response times from non-mainstream resolvers witin a client's geographic region may not matter.
Future work could measure page load times alongside DNS response times.  This
would provide greater insight into how user experience is directly impacted by
resolver performance.  Measurements of DoT resolvers could also be collected
to compare with DoH resolvers and understand if both are impacted by network
distance in the same way. 
