\abstract{

The ubiquitous presence of the Internet's Domain Name System (DNS) comes from the need for nearly all Internet services to translate easily readable domain names to their respective Internet Protocol (IP) addresses.
Servers on the Internet called DNS resolvers, traditionally managed by the user’s Internet Service Provider (ISP), perform this translation for users and applications worldwide.
Unencrypted DNS traffic between users and DNS resolvers can lead to privacy and security concerns.
In response to these privacy risks, two new protocols, DNS-over-HTTPS (DoH) and DNS-over-TLS (DoT), have been deployed to provide encrypted connections between users and DNS resolvers.
Fast response time for DNS queries is necessary to provide optimal user experience.

However, large client-side deployments of DoT and DoH have occurred in browsers that select between only a handful of resolvers.
If most users rely on the same popular encrypted DNS providers, those providers will have access to large amounts of data and user information, which shifts the original privacy concerns from ISPs to mainstream encrypted DNS resolvers.

In this paper we explore the performance of a large group of encrypted DNS resolvers supporting DoH by measuring DNS query response times from different global vantage points.
We find that most lesser-known resolvers have higher response times than well-known resolvers.
To our surprise, however, we find that some less popular resolvers perform comparably to (and sometimes better than) popular resolvers currently used by browsers.
Based on these results, users may be able to expand their options of resolvers beyond those that are well-known.
}
