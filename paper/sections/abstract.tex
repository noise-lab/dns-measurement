\abstract{

The ubiquitous presence of the Domain Name System (DNS) on the Internet comes from the need for nearly all Internet services to translate domain names that are easily readable by humans to their respective Internet Protocol (IP) addresses.
Servers on the Internet called DNS resolvers, traditionally managed by the user’s Internet Service Provider (ISP), perform this translation for users and applications worldwide.
Fast response time for DNS queries is necessary to provide optimal user experience.
Unencrypted DNS traffic between users and DNS resolvers can lead to privacy and security concerns.
In response to these privacy risks, two new protocols, DNS-over-HTTPS (DoH) and DNS-over-TLS (DoT), have been deployed to provide encrypted connections between users and DNS resolvers.

However, all current deployments of DoT and DoH have occurred in browsers that select between only a handful of resolvers.
If most users rely on the same mainstream encrypted DNS providers, those providers will have access to incredibly large amounts of data and user information, which simply shifts the original privacy concerns from ISPs to mainstream encrypted DNS resolvers.

In this paper we explore the reliability of a large group of encrypted DNS resolvers supporting DoH by measuring DNS query response times from different global vantage points.
We find that most lesser-known resolvers have higher response times than well-known resolvers.
To our surprise, however, we find that a select group of less popular resolvers operate better than mainstream resolvers.
Based on these results, users may be able to expand their options of trusted recursive resolvers beyond those that are well-known.