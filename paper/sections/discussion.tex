\section{Discussion}\label{sec:discussion}
Our results suggest that while clients may generally expect the best DoH query response times with current major resolvers, they may also use certain less-popular resolvers within their geographic region.
They also suggest that any resolver that is not majorly deployed is not useful if it is outside of one's region.
Of the \xxx{X} non-mainstream resolvers we measured, \xxx{X} performed within 10ms in median query response times for Clouflare, \xxx{X} for Google, and \xxx{X} for Quad9.
Furthermore, \xxx{X} of these non-mainstream resolvers were among the top five highest performing resolvers for a given vantage point.
Several of these resolvers were hosted by ISPs.
Thus, as more ISPs deploy DoH, we expect that clients will have even greater choice among which resolvers they use.

We note that previous work suggests DoH response times may not greatly matter for certain applications, such as web browsers.
Hounsel et al. found that while DoH response times were generally higher than traditional DNS, page load times were comparable~\cite{hounsel2020comparing}.
These measurements were performed with major DoH resolvers--namely Cloudflare, Google, and Quad9--but the results should generalize.
DoH connections based on HTTP/2 and above may utilize asynchronous queries, enabling the browser to send multiple DoH queries.
In short, if applications enable DoH queires to be asynchronously issued, and TCP/TLS sessions are re-used with long timeouts, then higher DNS response times from non-mainstream resolvers witin a client's geographic region may not matter.
