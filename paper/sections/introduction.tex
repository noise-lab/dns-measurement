\section{Introduction}\label{sec:intro}

The Domain Name System (DNS) is a critical component of the Internet's
infrastructure that translates human-readable domain names (\eg,
\texttt{google.com}) into Internet Protocol (IP) addresses~\cite{rfc1035}.
Most Internet communications begin with a client device sending DNS queries to
a recursive resolver, which in turn queries one or more name servers, which
ultimately refer the client to a server who can map the domain to an IP
address.
The \emph{response times} of these queries---the time 
to contact a recursive resolver, query various name servers, and return
the results---is important because the DNS underlies virtually all
communication on the Internet.  For example, loading a
web page, a browser must first resolve the domain names for each object on the
page before the objects themselves can be retrieved and rendered.  Thus, the
performance of DNS lookup is of utmost importance to application performance
such as web performance, as slow DNS lookup times will lead to slow web page
loads.

DNS did not originally take privacy and security into account: DNS queries
have historically been unencrypted, leaving users susceptible to
eavesdropping~\cite{dns-eavesdrop}; queries can also be intercepted and
manipulated to re-direct users to malicious websites~\cite{dns-redirect}.  To
address these types of privacy and security vulnerabilities, encrypted DNS
protocols have been developed and deployed, including DNS-over-HTTPS
(DoH)~\cite{rfc8484}, which is now deployed---and even enabled by default---in
many browsers.  DoH enables clients to communicate with recursive resolvers
over HTTPS, providing privacy and security guarantees that DNS previously
lacked.

For better or worse, most contemporary deployments of DoH have occurred in
browsers that provide limited options of
resolvers~\cite{ffChoices,chromeResolvers}.  If users continue to adopt
encrypted DNS and rely on a small set of recursive resolvers, these resolvers
will have increased visibility into data that was previously available to a
multitude of resolvers.  Such data includes queries that correspond to users'
web browsing histories.

\begin{table}[t]
    \centering
    \begin{tabular}{l|cccccc}
    \hline
    Browser & Cloudflare & Google & Quad9 & NextDNS & CleanBrowsing & OpenDNS
    \\
    \midrule
    Chrome    & \checkmark & \checkmark & & \checkmark & \checkmark & \checkmark \\
    Firefox  & \checkmark & & & \checkmark & & \\ 
    Edge   & \checkmark & \checkmark & \checkmark & \checkmark & \checkmark & \checkmark \\
    Opera            & \checkmark & \checkmark & & & & \\
    Brave            & \checkmark & \checkmark & \checkmark & \checkmark & \checkmark & \checkmark \\
    \bottomrule
    \end{tabular}
    \caption{Modern browsers provide only a few choices for encrypted DNS
    resolver, which we define as {\bf mainstream} resolvers.}
    \label{tab:SupportedResolvers}
\end{table}

\Fref{tab:SupportedResolvers} shows the DoH resolvers that have been deployed
to users of major browsers as of October 20,
2021~\cite{bravebrowser,edgebrowser,ffbrowser,chromebrowser,operabrowser}.  We
define the resolvers listed in~\Fref{tab:SupportedResolvers} as {\em
mainstream}.
Yet, many other DoH resolvers have been deployed that are currently
not in use by major browser deployments~\cite{dnscrypt}---in other words,
there are many non-mainstream DoH resolver deployments.  

Although DoH
addresses passive eavesdropping of DNS queries, it does not prevent resolvers
\emph{themselves} from seeing the contents of DNS queries.  Thus, some have
argued that major browser-based DoH deployments shift privacy concerns from
eavesdroppers to potential misuse by major DNS providers~\cite{vixie}.

Various previous studies have measured encrypted DNS performance, but they have mostly focused on mainstream DNS resolvers~\cite{borgolte2019dns,hounsel2020comparing,KResolver,lu2019end-to-end}.
In this paper, we expand on these previous studies, exploring the performance
of all encrypted DNS resolvers---from a variety of global (particularly
non-US) vantage points, as opposed to simply characterizing the mainstream DoH
providers from well-connected vantage points.
Towards this goal, we make the following contributions:
\begin{enumerate}
    \itemsep=-1pt
    \item We measure DoH response times a large list of resolvers, including
        both mainstream DoH resolvers that are included in major browser
        vendors {\em and} a large collection of non-mainstream resolvers.
%    \item We study whether users must use a major provider to experience acceptable DoH response times.
    \item We study how the performance of various DoH resolvers differ based
        on vantage point.
\end{enumerate}
\noindent
To our knowledge, this paper presents the first study of DoH performance
measurements for non-mainstream resolvers, as well as the first comparison of
DoH performance across a variety of vantage points, for a large number of
resolvers.
To perform these experiments, we developed and released an open-source
tool for measuring encrypted DNS performance to replicate and extend these
results, and to support further research on DoH performance.

The rest of the paper is organized as follows.  Section~\ref{sec:background}
provides background on DNS, including the origin of encrypted DNS and related
standards, and discusses related work.  Section~\ref{sec:method} details our research questions, the
experiments we conducted to study these questions, and the limitations of the
study.  Section~\ref{sec:results} presents the results of these experiments.
Section~\ref{sec:conclusion} concludes with a discussion of the implications
of these results and possible directions for future work. 

