\section{Introduction}\label{sec:intro}

The Domain Name System (DNS) was introduced in 1983 and constitutes a key aspect of the Internet[15].
On the Internet, computers communicate with each other using IP addresses.
The DNS translates human-readable domain names into these addresses.
Most Internet activity begins with a DNS query.
Multiple DNS queries are often involved in completing a lookup.
The entity performing this query and returning the result (converting a hostname such as \texttt{www.google.com} into a computer friendly IP address such as \texttt{192.168.1.1}) to the original computer is called a resolver, which is also known a recursive resolver or a recursor due to the  manner in which DNS servers communicate with each other to resolve a query.
The performance of the DNS query—namely how long it takes for the originating computer to contact the resolver, the resolver to do the translation from domain name to the IP address and return the result to the originating computer—is important because it can affect the overall experience of the user.
This time is called DNS query response time.

The DNS did not originally take user privacy and security into account.
Traditionally, DNS queries have been encrypted.
If DNS queries and responses are unencrypted, the websites a user is visiting can be revealed, which is a violation of their privacy.
Additionally, the information intercepted can be misused by bad actors to redirect users to malicious websites.
In order to address user privacy and security, DNS-over-TLS (DoT) and DNS-over-HTTPS (DoH) have been developed.
Both offer encryption of DNS data using the TLS protocol.
DoT uses a dedicated port (853)[8].
DoH, on the other hand, encapsulates DNS packets inside HTTPS (which uses TLS) and uses the same port as HTTPS (443)[7].

Most deployments of DoT and DoH have occurred in browsers that provide limited options of resolvers[9,6].
If most users adopt encrypted DNS and rely on the same major providers, those few providers will have increased visibility into user's browsing histories and access to data that was previously available to a multitude of resolvers”.

/////START TABLE
Major browsers including Brave, Opera, Google Chrome, Mozilla Firefox, and Microsoft Edge, only list Cloudflare, Cloudflare for Families, NextDNS, Google, Quad9, OpenDNS, and CleanBrowsing under their encrypted DNS provider options.
////END TABLE

The resolvers listed in Table 1 are currently being used by millions of browser users.
Many other DoH resolvers have been deployed[10].
The initial problem that gave reasons to make DNS traffic secure was that ISPs were able to log users' queries.
However, with browsers only supporting a limited number of major providers for encrypted DNS, the original problem seems to simply shift from ISPs to mainstream encrypted DNS providers.

In this paper, we begin to explore the usability of the DNS encrypted ecosystem as a whole rather than just major operators. We make the following contributions:
\begin{itemize}
\setlength\itemsep{0em}
\item We study the performance of encrypted DNS outside of the mainstream resolvers. We measure DNS query response times and latencies using a vast array of resolvers located across North America, Asia, Europe, and Australia. 
\item We study if users have a sufficient choice of trusted recursive resolvers beyond those of major providers and if one has to use a major provider to experience acceptable performance for DoH. 
\item We study how the performance of these servers differs depending on where an individual is on the network.
\item We make our modifications to a measurement tool openly available for further research to benchmark performance of DoH resolvers.
\end{itemize}

A number of previous studies have been performed comparing encrypted DNS performance with unencrypted DNS performance, but they have all focused on the mainstream DNS resolvers [few references].
This paper attempts at studying the performance of currently available DoH resolvers.

We begin by providing a historical background on DNS including the origin of encrypted DNS and related standards, followed by a description of the methods, measurements, and the setup of our experiments.
We then compare DNS query response times within major geographical areas based on the location of the resolvers—North America, Australia, Asia and Europe.
Finally, we discuss our results and provide direction for future research.