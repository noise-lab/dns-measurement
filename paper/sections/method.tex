\section{Method}\label{sec:method}

In this section, we describe the metrics used, how these metrics are measured, and our experiment setup.


\subsection{Metrics}
To understand how non-mainstream DoH resolvers function compared to mainstream resolvers, we measure DNS query response times and latency to these resolvers from multiple global vantage points.

\subsubsection{DNS Query Response Time}
DNS query response time is defined as the end-to-end time it takes for a user to initiate a query and receive a response.
We modify an existing custom tool to measure response times of Do53, DoT, and DoH queries in order to collect accurate DNS query response time measurements for DoH resolvers[5].
The tool was modified to allow continuous measurements of response times for DoH resolvers across multiple days.
With lists of specified recursors and domains, the tool prints performance data to a JSON file including information on the response time from each resolver to a domain and ping latencies to resolvers.
We chose three popular domains for our experiment: Google.com, a search engine, Netflix.com, a streaming service, and Facebook.com, a social media platform. 
We define a resolver as unreliable if it fails to respond.  

\subsubsection{Latency}
We measure latency to recursive resolvers by computing the average time it takes to receive an ICMP ping response.
We took an equal amount of ping measurements as response time measurements for a resolver.
To confirm our DNS response time measurements, we collected data on latency for each resolver. 

\subsection{Experiment Setup}
To provide a comparative assessment of DNS performance across DoH resolvers, we perform measurements across 75 DoH resolvers, grouped by their geographical locations—North America, Australia, Asia, and Europe[10].
We employed MaxMind's GeoLite2 databases to geolocate each DoH resolver[16].
That data was used to group the resolvers by location to make them more comparable. 

- Vantage points + description of hardware
- Network characteristics
- Description of tool
- Length of measurement
Our measurements were collected over the span of seven days. 
- Measurements performed on a Debian operating system 

\subsection{Limitations}
	- We don't measure page load times 	
	- We measure only a few domain names but that we don’t expect the performance trends to change
	- We performed measurements from EC2 instead of real users’ devices, but this may not matter because these resolvers are probably not all located next to data centers
